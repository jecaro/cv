\cvsection{Expérience}
\begin{cventries}
  \cventry
    {Ingénieur développement Cloud}
    {\href{https://www.linkedin.com/company/secucloud/}{Secucloud}}
    {Hambourg, Allemagne / Remote}
    {Depuis 11/2020}
    {
      \begin{cvitems}
        \item {Développement d'outils de sécurité réseau pour le Cloud}
        \item {Equipe internationale en 100\% remote}
        \item {Environnement technique: Linux, Haskell, JavaScript, nix,
            Ansible, Clickhouse}
      \end{cvitems}
    }

  \cventry
    {Enseignant programmation fonctionnelle}
    {\href{https://www.epitech.eu/}{Epitech}}
    {Rennes}
    {09/2021 - 06/2023 (temps partiel)}
    {
      \begin{cvitems}
          \item {Soutien aux étudiants dans leurs projets Haskell (Evaluateur
              d'expressions arithmétiques, interpréteur Lisp, algorithme de
              compression d'images~...)}
          \item {Environnement technique: Linux, Haskell, nix}
      \end{cvitems}
    }

  \cventry
    {Ingénieur d'étude en société de services}
    {\href{https://www.scalian.com/}{Scalian}}
    {Rennes}
    {04/2007 - 10/2020}
    {
      \begin{cvitems}
        \item {Développement d'outils de simulation au forfait au sein du
            département Simulation, HPC \& Réalité virtuelle}
        \item {Architecture et responsable technique}
        \item {Rédaction de documentation technique}
        \item {Clients: DGA, ONERA, CEA, CNES, IRSN}
        \item {Environnement technique: Windows/Linux, C++, Python, Matlab, Qt,
            openscenegraph, gdal}
      \end{cvitems}
    }

  \cventry
    {Ingénieur de recherche}
    {\href{https://www.labri.fr/}{LaBRI} équipe Analyse et indexation vidéo}
    {Bordeaux}
    {07/2006 – 08/2006}
    {
      \begin{cvitems}
        \item {Préparation des concours d'indexation vidéo TRECVID et ARGOS}
        \item {Administration système}
        \item {Environnement technique: Linux, C/C++, XML, Perl, Bash}
      \end{cvitems}
    }

  \cventry
    {Ingénieur de recherche}
    {\href{https://www.inria.fr/}{INRIA} équipe IPARLA}
    {Bordeaux}
    {09/2004 – 04/2006}
    {
      \begin{cvitems}
        \item {Visualisation d'environnements urbains sur terminaux mobiles en
            utilisant des techniques de rendu non photo-réaliste}
        \item {Rédaction et présentation d'un article scientifique à la
            conférence Web3D 2006 à Washington}
        \item {Environnement technique: Linux, Windows, Windows Mobile, C/C++,
            Bash, Perl, VRML97, X3D}
      \end{cvitems}
    }

  \cventry
    {Ingénieur développement}
    {\href{https://www.keyghost.com/}{Keyghost}}
    {Christchurch, Nouvelle-Zélande}
    {04/2003 - 12/2003}
    {
      \begin{cvitems}
        \item {Développement d'applications de sécurité}
		\item {Environnement technique: Windows, Delphi, C/C++, Octave}
      \end{cvitems}
    }

  \cventry
    {Stage suivi d'un contrat d'ingénieur de recherche}
    {Visualpix}
    {Bordeaux}
    {04/2002 - 12/2002}
    {
      \begin{cvitems}
        \item {Amélioration d'un codec de compression vidéo}
        \item {Environnement technique: Linux, C}
      \end{cvitems}
    }

  \cventry
    {Backend Développeur, Administrateur système}
    {Glisshouse}
    {Marseille}
    {07/2000 - 03/2001}
    {
      \begin{cvitems}
        \item {Web: Maintenance d'un magasin en ligne, développement de CMS}
        \item {Réseau: Installation, configuration et administration du réseau
            local (routeur, DNS, firewall, proxy, serveur de fichiers,
            d'impression)}
        \item {Environnement technique: Linux, Php, MySQL, Apache, Samba, cron}
      \end{cvitems}
    }

\end{cventries}
